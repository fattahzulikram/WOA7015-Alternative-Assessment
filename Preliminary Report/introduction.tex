\section{Introduction}

\textbf{Medical Visual Question Answering (Med-VQA)} combines medical artificial intelligence with visual question answering, where a
system predicts plausible answers to clinically relevant questions in natural language given medical images. Unlike general VQA systems,
Med-VQA is considered more challenging for different factors, including the difficulty in creating large datasets with manually 
annotated question-answer pairs, the need for sophisticated models that can focus on fine-grained details of medical images, and the 
necessity to train a model with medical contexts to properly correlate the questions to the images.~\autocite{Lin_2023} Although the 
research of Med-VQA is still relatively new \autocite{WOS:000786144100348}, there are some publicly available datasets that 
are well-annotated for training, validating, and testing Med-VQA systems, including VQA-RAD \autocite{vqarad}, PathVQA \autocite{pathvqa}, 
and SLAKE \autocite{WOS:000786144100348}.

Traditionally, research on Med-VQA employed discriminative models that focused on image recognition and classification where models were 
trained to identify specific features within medical images. These architectures commonly utilize CNNs as image encoders
and RNN/LSTM as the text encoder.~\autocite{bioengineering10030380} However, these systems typically relied on predefined categories, and
while they were effective in closed—set scenarios, they lacked the ability to answer complex, natural language questions that could 
guide clinical decision-making.~\autocite{BENCHAABANE2025200545}

With the recent advances in \textbf{Vision-Language Models (VLM)}, the dynamic has shifted towards fine-tuning powerful foundational 
models that are pre-trained with large-scale databases of image-text pairs. Models such as BERT, CLIP, BLIP, LLaVA, and others have achieved 
remarkable results on different VQA datasets. For specific domains, domain-specific knowledge is injected within the models, resulting 
in domain-specific models like BioMedBLIP~\autocite{biomedblip}, BioMedCLIP~\autocite{zhang2025biomedclipmultimodalbiomedicalfoundation}, 
and so on. 

With all these advances, it becomes a necessity to benchmark them to understand the trade-offs between traditional, discriminative 
models versus modern, pre-trained vision language models. These comparisons can help identify the future direction that can be taken to 
advance medical VQA systems further.